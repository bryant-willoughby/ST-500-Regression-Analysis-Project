% Options for packages loaded elsewhere
\PassOptionsToPackage{unicode}{hyperref}
\PassOptionsToPackage{hyphens}{url}
\documentclass[
]{article}
\usepackage{xcolor}
\usepackage[margin=1in]{geometry}
\usepackage{amsmath,amssymb}
\setcounter{secnumdepth}{-\maxdimen} % remove section numbering
\usepackage{iftex}
\ifPDFTeX
  \usepackage[T1]{fontenc}
  \usepackage[utf8]{inputenc}
  \usepackage{textcomp} % provide euro and other symbols
\else % if luatex or xetex
  \usepackage{unicode-math} % this also loads fontspec
  \defaultfontfeatures{Scale=MatchLowercase}
  \defaultfontfeatures[\rmfamily]{Ligatures=TeX,Scale=1}
\fi
\usepackage{lmodern}
\ifPDFTeX\else
  % xetex/luatex font selection
\fi
% Use upquote if available, for straight quotes in verbatim environments
\IfFileExists{upquote.sty}{\usepackage{upquote}}{}
\IfFileExists{microtype.sty}{% use microtype if available
  \usepackage[]{microtype}
  \UseMicrotypeSet[protrusion]{basicmath} % disable protrusion for tt fonts
}{}
\makeatletter
\@ifundefined{KOMAClassName}{% if non-KOMA class
  \IfFileExists{parskip.sty}{%
    \usepackage{parskip}
  }{% else
    \setlength{\parindent}{0pt}
    \setlength{\parskip}{6pt plus 2pt minus 1pt}}
}{% if KOMA class
  \KOMAoptions{parskip=half}}
\makeatother
\usepackage{color}
\usepackage{fancyvrb}
\newcommand{\VerbBar}{|}
\newcommand{\VERB}{\Verb[commandchars=\\\{\}]}
\DefineVerbatimEnvironment{Highlighting}{Verbatim}{commandchars=\\\{\}}
% Add ',fontsize=\small' for more characters per line
\usepackage{framed}
\definecolor{shadecolor}{RGB}{248,248,248}
\newenvironment{Shaded}{\begin{snugshade}}{\end{snugshade}}
\newcommand{\AlertTok}[1]{\textcolor[rgb]{0.94,0.16,0.16}{#1}}
\newcommand{\AnnotationTok}[1]{\textcolor[rgb]{0.56,0.35,0.01}{\textbf{\textit{#1}}}}
\newcommand{\AttributeTok}[1]{\textcolor[rgb]{0.13,0.29,0.53}{#1}}
\newcommand{\BaseNTok}[1]{\textcolor[rgb]{0.00,0.00,0.81}{#1}}
\newcommand{\BuiltInTok}[1]{#1}
\newcommand{\CharTok}[1]{\textcolor[rgb]{0.31,0.60,0.02}{#1}}
\newcommand{\CommentTok}[1]{\textcolor[rgb]{0.56,0.35,0.01}{\textit{#1}}}
\newcommand{\CommentVarTok}[1]{\textcolor[rgb]{0.56,0.35,0.01}{\textbf{\textit{#1}}}}
\newcommand{\ConstantTok}[1]{\textcolor[rgb]{0.56,0.35,0.01}{#1}}
\newcommand{\ControlFlowTok}[1]{\textcolor[rgb]{0.13,0.29,0.53}{\textbf{#1}}}
\newcommand{\DataTypeTok}[1]{\textcolor[rgb]{0.13,0.29,0.53}{#1}}
\newcommand{\DecValTok}[1]{\textcolor[rgb]{0.00,0.00,0.81}{#1}}
\newcommand{\DocumentationTok}[1]{\textcolor[rgb]{0.56,0.35,0.01}{\textbf{\textit{#1}}}}
\newcommand{\ErrorTok}[1]{\textcolor[rgb]{0.64,0.00,0.00}{\textbf{#1}}}
\newcommand{\ExtensionTok}[1]{#1}
\newcommand{\FloatTok}[1]{\textcolor[rgb]{0.00,0.00,0.81}{#1}}
\newcommand{\FunctionTok}[1]{\textcolor[rgb]{0.13,0.29,0.53}{\textbf{#1}}}
\newcommand{\ImportTok}[1]{#1}
\newcommand{\InformationTok}[1]{\textcolor[rgb]{0.56,0.35,0.01}{\textbf{\textit{#1}}}}
\newcommand{\KeywordTok}[1]{\textcolor[rgb]{0.13,0.29,0.53}{\textbf{#1}}}
\newcommand{\NormalTok}[1]{#1}
\newcommand{\OperatorTok}[1]{\textcolor[rgb]{0.81,0.36,0.00}{\textbf{#1}}}
\newcommand{\OtherTok}[1]{\textcolor[rgb]{0.56,0.35,0.01}{#1}}
\newcommand{\PreprocessorTok}[1]{\textcolor[rgb]{0.56,0.35,0.01}{\textit{#1}}}
\newcommand{\RegionMarkerTok}[1]{#1}
\newcommand{\SpecialCharTok}[1]{\textcolor[rgb]{0.81,0.36,0.00}{\textbf{#1}}}
\newcommand{\SpecialStringTok}[1]{\textcolor[rgb]{0.31,0.60,0.02}{#1}}
\newcommand{\StringTok}[1]{\textcolor[rgb]{0.31,0.60,0.02}{#1}}
\newcommand{\VariableTok}[1]{\textcolor[rgb]{0.00,0.00,0.00}{#1}}
\newcommand{\VerbatimStringTok}[1]{\textcolor[rgb]{0.31,0.60,0.02}{#1}}
\newcommand{\WarningTok}[1]{\textcolor[rgb]{0.56,0.35,0.01}{\textbf{\textit{#1}}}}
\usepackage{graphicx}
\makeatletter
\newsavebox\pandoc@box
\newcommand*\pandocbounded[1]{% scales image to fit in text height/width
  \sbox\pandoc@box{#1}%
  \Gscale@div\@tempa{\textheight}{\dimexpr\ht\pandoc@box+\dp\pandoc@box\relax}%
  \Gscale@div\@tempb{\linewidth}{\wd\pandoc@box}%
  \ifdim\@tempb\p@<\@tempa\p@\let\@tempa\@tempb\fi% select the smaller of both
  \ifdim\@tempa\p@<\p@\scalebox{\@tempa}{\usebox\pandoc@box}%
  \else\usebox{\pandoc@box}%
  \fi%
}
% Set default figure placement to htbp
\def\fps@figure{htbp}
\makeatother
\setlength{\emergencystretch}{3em} % prevent overfull lines
\providecommand{\tightlist}{%
  \setlength{\itemsep}{0pt}\setlength{\parskip}{0pt}}
\usepackage{bookmark}
\IfFileExists{xurl.sty}{\usepackage{xurl}}{} % add URL line breaks if available
\urlstyle{same}
\hypersetup{
  pdftitle={WilloughbyHW4},
  pdfauthor={Bryant Willoughby},
  hidelinks,
  pdfcreator={LaTeX via pandoc}}

\title{WilloughbyHW4}
\author{Bryant Willoughby}
\date{2025-10-09}

\begin{document}
\maketitle

\section{Homework 4}\label{homework-4}

Note: Display only the plots that are relevant to the questions. Present
your diagnostics in a logical order.

\begin{Shaded}
\begin{Highlighting}[]
\CommentTok{\# Load data library}
\FunctionTok{library}\NormalTok{(faraway)}
\end{Highlighting}
\end{Shaded}

\begin{verbatim}
## Warning: package 'faraway' was built under R version 4.4.3
\end{verbatim}

\section{Load the Data and Fit
Models}\label{load-the-data-and-fit-models}

\texttt{teengamb} contains information about a study of teenage gambling
in Britain. It contains 47 rows and 5 columns. To predict expenditures
from all other available variables, we fit the following two linear
regressions:

\[ \text{response } Y_{org}:= gamble \sim sex + status + income + verbal \\  \text{response } log(Y_{org}) := log(gamble + 1) \sim  sex + status + income + verbal\]

\begin{Shaded}
\begin{Highlighting}[]
\CommentTok{\# help(teengamb)}
\FunctionTok{data}\NormalTok{(teengamb)}
\FunctionTok{head}\NormalTok{(teengamb)}
\end{Highlighting}
\end{Shaded}

\begin{verbatim}
##   sex status income verbal gamble
## 1   1     51   2.00      8    0.0
## 2   1     28   2.50      8    0.0
## 3   1     37   2.00      6    0.0
## 4   1     28   7.00      4    7.3
## 5   1     65   2.00      8   19.6
## 6   1     61   3.47      6    0.1
\end{verbatim}

\begin{Shaded}
\begin{Highlighting}[]
\NormalTok{fit1 }\OtherTok{\textless{}{-}} \FunctionTok{lm}\NormalTok{(gamble }\SpecialCharTok{\textasciitilde{}}\NormalTok{ sex }\SpecialCharTok{+}\NormalTok{ status }\SpecialCharTok{+}\NormalTok{ income }\SpecialCharTok{+}\NormalTok{ verbal, }\AttributeTok{data =}\NormalTok{ teengamb)}
\FunctionTok{summary}\NormalTok{(fit1)}
\end{Highlighting}
\end{Shaded}

\begin{verbatim}
## 
## Call:
## lm(formula = gamble ~ sex + status + income + verbal, data = teengamb)
## 
## Residuals:
##     Min      1Q  Median      3Q     Max 
## -51.082 -11.320  -1.451   9.452  94.252 
## 
## Coefficients:
##              Estimate Std. Error t value Pr(>|t|)    
## (Intercept)  22.55565   17.19680   1.312   0.1968    
## sex         -22.11833    8.21111  -2.694   0.0101 *  
## status        0.05223    0.28111   0.186   0.8535    
## income        4.96198    1.02539   4.839 1.79e-05 ***
## verbal       -2.95949    2.17215  -1.362   0.1803    
## ---
## Signif. codes:  0 '***' 0.001 '**' 0.01 '*' 0.05 '.' 0.1 ' ' 1
## 
## Residual standard error: 22.69 on 42 degrees of freedom
## Multiple R-squared:  0.5267, Adjusted R-squared:  0.4816 
## F-statistic: 11.69 on 4 and 42 DF,  p-value: 1.815e-06
\end{verbatim}

\begin{Shaded}
\begin{Highlighting}[]
\NormalTok{fit2 }\OtherTok{\textless{}{-}} \FunctionTok{lm}\NormalTok{(}\FunctionTok{log}\NormalTok{(gamble }\SpecialCharTok{+} \DecValTok{1}\NormalTok{) }\SpecialCharTok{\textasciitilde{}}\NormalTok{ sex }\SpecialCharTok{+}\NormalTok{ status }\SpecialCharTok{+}\NormalTok{ income }\SpecialCharTok{+}\NormalTok{ verbal, }\AttributeTok{data =}\NormalTok{ teengamb)}
\FunctionTok{summary}\NormalTok{(fit2)}
\end{Highlighting}
\end{Shaded}

\begin{verbatim}
## 
## Call:
## lm(formula = log(gamble + 1) ~ sex + status + income + verbal, 
##     data = teengamb)
## 
## Residuals:
##      Min       1Q   Median       3Q      Max 
## -2.35012 -0.56865  0.00413  0.71512  1.90319 
## 
## Coefficients:
##             Estimate Std. Error t value Pr(>|t|)    
## (Intercept)  1.71620    0.82240   2.087   0.0430 *  
## sex         -0.87120    0.39268  -2.219   0.0320 *  
## status       0.02983    0.01344   2.219   0.0320 *  
## income       0.21565    0.04904   4.398 7.33e-05 ***
## verbal      -0.26165    0.10388  -2.519   0.0157 *  
## ---
## Signif. codes:  0 '***' 0.001 '**' 0.01 '*' 0.05 '.' 0.1 ' ' 1
## 
## Residual standard error: 1.085 on 42 degrees of freedom
## Multiple R-squared:  0.5206, Adjusted R-squared:  0.475 
## F-statistic:  11.4 on 4 and 42 DF,  p-value: 2.347e-06
\end{verbatim}

\section{Problem 1}\label{problem-1}

Perform regression diagnostics on these two models and compare the
answers to the following questions based on the two models.

For models 1 and 2, we assume the following:

\[\epsilon \stackrel{\text{iid}}{\sim} N(0,\sigma^2I) \Leftrightarrow \\ i) \text{ }E(\epsilon|X) = 0 \\
ii) \text{ }Var(\epsilon|X) = \sigma^2I \\ iii) \text{ } \epsilon's \text{ are independent and identically distributed}\]

For this problem, I check conditions i) and ii) with the following
procedures:

\begin{itemize}
\tightlist
\item
  check the existence of patterns in the residuals against fitted
  values. Are the residuals centered at 0 in all areas of fitted values?
\item
  Check the constant variance assumption for the errors
\end{itemize}

For model 1, the residuals vs.~fitted plot shows a fanning effect where
residuals are more spread out about zero for larger fitted values.

To check linearity, and hence the zero mean assumption for the
residuals, I fit the following model:
\(\hat{\epsilon} \sim \hat{Y} + \hat{Y^2}\). The linear
(\(p \approx 0.03\)) and quadratic (\(p \approx 0.02\)) term are both
significant. That is, there is evidence to suggest deviation from the
linearity assumption in favor of a higher-order (quadratic) effect.

To check constant the constant error variance assumption, I fit the
following model: \(|\hat{\epsilon}| \sim \hat{Y}\). The significant
slope term (\(p \approx 0.009\)) provides evidence against constant
variance in favor of heteroscedasticity.

\begin{Shaded}
\begin{Highlighting}[]
\CommentTok{\# linearity \& constant variance checking (model 1)}
\NormalTok{fit.vals }\OtherTok{=}\NormalTok{ fit1}\SpecialCharTok{$}\NormalTok{fitted }
\NormalTok{res.vals }\OtherTok{=}\NormalTok{ fit1}\SpecialCharTok{$}\NormalTok{residual }


\CommentTok{\#residuals vs fitted values }
\FunctionTok{plot}\NormalTok{(fit.vals, res.vals, }\AttributeTok{main =} \StringTok{"Residuals vs. Fitted Plot"}\NormalTok{, }
       \AttributeTok{xlab =} \StringTok{"Fitted"}\NormalTok{, }\AttributeTok{ylab =} \StringTok{"Residuals"}\NormalTok{)}
\FunctionTok{abline}\NormalTok{(}\AttributeTok{h =} \DecValTok{0}\NormalTok{)}
\end{Highlighting}
\end{Shaded}

\pandocbounded{\includegraphics[keepaspectratio]{WilloughbyHW4_files/figure-latex/unnamed-chunk-5-1.pdf}}

\begin{Shaded}
\begin{Highlighting}[]
\FunctionTok{paste}\NormalTok{(}\StringTok{"Linearity (and hence residual mean zero) assumption checking:"}\NormalTok{)}
\end{Highlighting}
\end{Shaded}

\begin{verbatim}
## [1] "Linearity (and hence residual mean zero) assumption checking:"
\end{verbatim}

\begin{Shaded}
\begin{Highlighting}[]
\FunctionTok{summary}\NormalTok{(}\FunctionTok{lm}\NormalTok{(res.vals }\SpecialCharTok{\textasciitilde{}}\NormalTok{ fit.vals }\SpecialCharTok{+} \FunctionTok{I}\NormalTok{(fit.vals}\SpecialCharTok{\^{}}\DecValTok{2}\NormalTok{)))}
\end{Highlighting}
\end{Shaded}

\begin{verbatim}
## 
## Call:
## lm(formula = res.vals ~ fit.vals + I(fit.vals^2))
## 
## Residuals:
##     Min      1Q  Median      3Q     Max 
## -51.959  -7.639  -2.795   3.473  90.186 
## 
## Coefficients:
##                Estimate Std. Error t value Pr(>|t|)  
## (Intercept)    3.422476   4.204807   0.814   0.4201  
## fit.vals      -0.717893   0.316388  -2.269   0.0282 *
## I(fit.vals^2)  0.011795   0.004711   2.504   0.0161 *
## ---
## Signif. codes:  0 '***' 0.001 '**' 0.01 '*' 0.05 '.' 0.1 ' ' 1
## 
## Residual standard error: 20.74 on 44 degrees of freedom
## Multiple R-squared:  0.1247, Adjusted R-squared:  0.0849 
## F-statistic: 3.134 on 2 and 44 DF,  p-value: 0.05341
\end{verbatim}

\begin{Shaded}
\begin{Highlighting}[]
\FunctionTok{paste}\NormalTok{(}\StringTok{"Constant error variance assumption checking:"}\NormalTok{)}
\end{Highlighting}
\end{Shaded}

\begin{verbatim}
## [1] "Constant error variance assumption checking:"
\end{verbatim}

\begin{Shaded}
\begin{Highlighting}[]
\FunctionTok{summary}\NormalTok{(}\FunctionTok{lm}\NormalTok{(}\FunctionTok{abs}\NormalTok{(res.vals) }\SpecialCharTok{\textasciitilde{}}\NormalTok{ fit.vals))}
\end{Highlighting}
\end{Shaded}

\begin{verbatim}
## 
## Call:
## lm(formula = abs(res.vals) ~ fit.vals)
## 
## Residuals:
##     Min      1Q  Median      3Q     Max 
## -24.944  -7.411  -2.467   3.409  68.590 
## 
## Coefficients:
##             Estimate Std. Error t value Pr(>|t|)   
## (Intercept)   9.3303     2.8789   3.241  0.00224 **
## fit.vals      0.2645     0.0968   2.732  0.00895 **
## ---
## Signif. codes:  0 '***' 0.001 '**' 0.01 '*' 0.05 '.' 0.1 ' ' 1
## 
## Residual standard error: 15.02 on 45 degrees of freedom
## Multiple R-squared:  0.1423, Adjusted R-squared:  0.1232 
## F-statistic: 7.466 on 1 and 45 DF,  p-value: 0.008952
\end{verbatim}

For model 2, the residuals vs.~fitted plot shows an even distribution
(scatter) of residuals according to the range of fitted values.

To check linearity, and hence the zero mean assumption for the
residuals, I fit the following model:
\(\hat{\epsilon} \sim \hat{Y} + \hat{Y^2}\). Since the quadratic term is
insignificant (\(p \approx 0.66\)) there is not evidence to suggest
deviation from the linearity assumption in favor of a higher-order
(quadratic) effect.

To check constant the constant error variance assumption, I fit the
following model: \(|\hat{\epsilon}| \sim \hat{Y}\). The insignificant
slope term (\(p \approx 0.97\)) does not provide evidence to suggest
non-constant variance.

\begin{Shaded}
\begin{Highlighting}[]
\CommentTok{\# constant variance checking (model 2)}
\NormalTok{fit.vals }\OtherTok{=}\NormalTok{ fit2}\SpecialCharTok{$}\NormalTok{fitted }
\NormalTok{res.vals }\OtherTok{=}\NormalTok{ fit2}\SpecialCharTok{$}\NormalTok{residual }


\CommentTok{\#residuals vs fitted values }
\FunctionTok{plot}\NormalTok{(fit.vals, res.vals, }\AttributeTok{main =} \StringTok{"Residuals vs. Fitted Plot"}\NormalTok{, }
       \AttributeTok{xlab =} \StringTok{"Fitted"}\NormalTok{, }\AttributeTok{ylab =} \StringTok{"Residuals"}\NormalTok{)}
\FunctionTok{abline}\NormalTok{(}\AttributeTok{h =} \DecValTok{0}\NormalTok{)}
\end{Highlighting}
\end{Shaded}

\pandocbounded{\includegraphics[keepaspectratio]{WilloughbyHW4_files/figure-latex/unnamed-chunk-6-1.pdf}}

\begin{Shaded}
\begin{Highlighting}[]
\FunctionTok{paste}\NormalTok{(}\StringTok{"Linearity (and hence residual mean zero) assumption checking:"}\NormalTok{)}
\end{Highlighting}
\end{Shaded}

\begin{verbatim}
## [1] "Linearity (and hence residual mean zero) assumption checking:"
\end{verbatim}

\begin{Shaded}
\begin{Highlighting}[]
\FunctionTok{summary}\NormalTok{(}\FunctionTok{lm}\NormalTok{(res.vals }\SpecialCharTok{\textasciitilde{}}\NormalTok{ fit.vals }\SpecialCharTok{+} \FunctionTok{I}\NormalTok{(fit.vals}\SpecialCharTok{\^{}}\DecValTok{2}\NormalTok{)))}
\end{Highlighting}
\end{Shaded}

\begin{verbatim}
## 
## Call:
## lm(formula = res.vals ~ fit.vals + I(fit.vals^2))
## 
## Residuals:
##      Min       1Q   Median       3Q      Max 
## -2.29195 -0.56134  0.01941  0.74038  1.90835 
## 
## Coefficients:
##               Estimate Std. Error t value Pr(>|t|)
## (Intercept)    0.18893    0.54074   0.349    0.728
## fit.vals      -0.22175    0.52842  -0.420    0.677
## I(fit.vals^2)  0.04936    0.11314   0.436    0.665
## 
## Residual standard error: 1.058 on 44 degrees of freedom
## Multiple R-squared:  0.004306,   Adjusted R-squared:  -0.04095 
## F-statistic: 0.09515 on 2 and 44 DF,  p-value: 0.9094
\end{verbatim}

\begin{Shaded}
\begin{Highlighting}[]
\FunctionTok{paste}\NormalTok{(}\StringTok{"Constant error variance assumption checking:"}\NormalTok{)}
\end{Highlighting}
\end{Shaded}

\begin{verbatim}
## [1] "Constant error variance assumption checking:"
\end{verbatim}

\begin{Shaded}
\begin{Highlighting}[]
\FunctionTok{summary}\NormalTok{(}\FunctionTok{lm}\NormalTok{(}\FunctionTok{abs}\NormalTok{(res.vals) }\SpecialCharTok{\textasciitilde{}}\NormalTok{ fit.vals))}
\end{Highlighting}
\end{Shaded}

\begin{verbatim}
## 
## Call:
## lm(formula = abs(res.vals) ~ fit.vals)
## 
## Residuals:
##     Min      1Q  Median      3Q     Max 
## -0.7875 -0.5635 -0.1763  0.5921  1.5537 
## 
## Coefficients:
##             Estimate Std. Error t value Pr(>|t|)    
## (Intercept) 0.788325   0.202794   3.887 0.000331 ***
## fit.vals    0.003324   0.090409   0.037 0.970830    
## ---
## Signif. codes:  0 '***' 0.001 '**' 0.01 '*' 0.05 '.' 0.1 ' ' 1
## 
## Residual standard error: 0.6626 on 45 degrees of freedom
## Multiple R-squared:  3.005e-05,  Adjusted R-squared:  -0.02219 
## F-statistic: 0.001352 on 1 and 45 DF,  p-value: 0.9708
\end{verbatim}

\section{Problem 2}\label{problem-2}

Now focus on the regression model with the response
\(log(Y_{org}) = log(gamble + 1)\)

\begin{itemize}
\tightlist
\item
  check the normality assumption
\item
  check for outliers
\end{itemize}

I first plotted a histogram of the residuals, overlaying a density
curve. It is unimodal with a slight left-skew. Additional checking is
required for a conclusive determination about normality of the errors.
Next, I plot the sorted residuals: \(\hat{\epsilon}_{[i]}\) against the
theoretical normal quantiles: \(u_i = \Phi^{-1}(\frac{i}{n+1})\) for
\(i = 1, \ldots, n=47\). There appears to be a small, random deviation
between these quantities, as seen by the points following an
approximately linear trend.

The Shapiro-Wilk test for normality assumes the following hypotheses:
\(H_o: \epsilon_i \text{ are normally distributed}\) versus
\(H_a: \epsilon_i \text{ are not normally distributed}\). Given the
non-extreme test statistic (\(w \approx 0.98\)) and sufficiently large
p-value (\(p \approx 0.44\)), there is evidence to suggest a violation
of the normality condition. Note that for the Shapiro-Wilk test,
concerns about statistical power and its sensitivity to sample size
motivate checking normality through additional methods, as done above.

While the conclusions are contradictory, I would argue that the
normality assumption is approximately met, as described by my
observations regarding the histogram of residuals and normal QQ-plot.

\begin{Shaded}
\begin{Highlighting}[]
\CommentTok{\# normality assumption }

\CommentTok{\#histogram of residuals }
\FunctionTok{hist}\NormalTok{(fit2}\SpecialCharTok{$}\NormalTok{residual, }\AttributeTok{freq =}\NormalTok{ F, }\AttributeTok{main =} \StringTok{"Histogram of Residuals"}\NormalTok{, }\AttributeTok{xlab =} \StringTok{"Resdiuals"}\NormalTok{)}
\FunctionTok{lines}\NormalTok{(}\FunctionTok{density}\NormalTok{(fit2}\SpecialCharTok{$}\NormalTok{residual))}
\end{Highlighting}
\end{Shaded}

\pandocbounded{\includegraphics[keepaspectratio]{WilloughbyHW4_files/figure-latex/unnamed-chunk-7-1.pdf}}

\begin{Shaded}
\begin{Highlighting}[]
\CommentTok{\#qq{-}plot}
\FunctionTok{qqnorm}\NormalTok{(fit2}\SpecialCharTok{$}\NormalTok{residual, }\AttributeTok{ylab =} \StringTok{"Residuals"}\NormalTok{)}
\FunctionTok{qqline}\NormalTok{(fit2}\SpecialCharTok{$}\NormalTok{residual)}
\end{Highlighting}
\end{Shaded}

\pandocbounded{\includegraphics[keepaspectratio]{WilloughbyHW4_files/figure-latex/unnamed-chunk-7-2.pdf}}

\begin{Shaded}
\begin{Highlighting}[]
\CommentTok{\#shapiro{-}wilk test }
\FunctionTok{shapiro.test}\NormalTok{(fit2}\SpecialCharTok{$}\NormalTok{residual)}
\end{Highlighting}
\end{Shaded}

\begin{verbatim}
## 
##  Shapiro-Wilk normality test
## 
## data:  fit2$residual
## W = 0.97609, p-value = 0.4418
\end{verbatim}

Next, I check for outliers. Outliers are unusual points that do not fit
the model well. To distinguish between outliers and large residuals, we
exclude point i, recompute \(\hat{\beta}_{(i)}\) and hence
\(\hat{y}_{(i)}\hat{X_i^T\hat{\beta}}_{(i)}\). We conclude \(i\) is an
outlier if \(|y_i = \hat{y}_{(i)}|\) is `large.' To quantify `large', we
use the (externally) studentized residuals as shown below, with the
derivations taken to be true:

\[\begin{align*} \text{ Externally Studentized Residuals: } t_i &:= \frac{y_i - \hat{y}_{(i)}}{\hat{\sigma}_{(i)}\sqrt{1 + X_i^T(X_{(i)}^TX_{(i)})^{-1}X_i}} \\ &= r_i (\frac{n - (p + 1) - 1}{n - (p + 1) - r_i^2})^{1/2} \\ &\text{ where } r_i = \frac{\hat{\epsilon_i}}{\hat{\sigma}\sqrt{1 - h_i}} \\ &\text{ Note that } t_i \sim t_{n - (p + 1) - 1} \text{ under } H_o: i_{th} \text{ observation not an outlier} \end{align*}\]

First, I plot all \(t_i, i \in {1, \ldots, 47}\) against the respective
observations. I flag any values for which \(|t_i| \ge 1.75\). The five
observations which meet this threshold are highlighted in blue. They
represent the most extreme \(t_i\) values; in other words, these are the
observations with the most extreme deviations between \(y_i\) and
\(\hat{y}_{(i)}\). These are potential outliers to explore further.

\begin{Shaded}
\begin{Highlighting}[]
\NormalTok{ti }\OtherTok{\textless{}{-}} \FunctionTok{rstudent}\NormalTok{(fit2)}
\NormalTok{thr }\OtherTok{\textless{}{-}} \FloatTok{1.75}

\FunctionTok{plot}\NormalTok{(}
\NormalTok{  ti,}
  \AttributeTok{pch =} \DecValTok{19}\NormalTok{, }\AttributeTok{col =} \StringTok{"gray30"}\NormalTok{,}
  \AttributeTok{main =} \StringTok{"Externally Studentized Residuals"}\NormalTok{,}
  \AttributeTok{xlab =} \StringTok{"Observation index"}\NormalTok{,}
  \AttributeTok{ylab =} \StringTok{"Externally studentized residual"}\NormalTok{, }
  \AttributeTok{ylim =} \FunctionTok{c}\NormalTok{(}\SpecialCharTok{{-}}\DecValTok{3}\NormalTok{,}\DecValTok{3}\NormalTok{)}
\NormalTok{)}
\FunctionTok{abline}\NormalTok{(}\AttributeTok{h =} \FunctionTok{c}\NormalTok{(}\SpecialCharTok{{-}}\NormalTok{thr, thr), }\AttributeTok{col =} \StringTok{"red"}\NormalTok{, }\AttributeTok{lty =} \DecValTok{2}\NormalTok{, }\AttributeTok{lwd =} \DecValTok{1}\NormalTok{)}

\NormalTok{idx }\OtherTok{\textless{}{-}} \FunctionTok{which}\NormalTok{(}\FunctionTok{abs}\NormalTok{(ti) }\SpecialCharTok{\textgreater{}=}\NormalTok{ thr)}
\FunctionTok{points}\NormalTok{(idx, ti[idx], }\AttributeTok{pch =} \DecValTok{19}\NormalTok{, }\AttributeTok{col =} \StringTok{"blue"}\NormalTok{)}
\NormalTok{pos }\OtherTok{\textless{}{-}} \FunctionTok{ifelse}\NormalTok{(ti[idx] }\SpecialCharTok{\textgreater{}=} \DecValTok{0}\NormalTok{, }\DecValTok{3}\NormalTok{, }\DecValTok{1}\NormalTok{)  }\CommentTok{\# label above if positive, below if negative}
\FunctionTok{text}\NormalTok{(idx, ti[idx], }\AttributeTok{labels =}\NormalTok{ idx, }\AttributeTok{pos =}\NormalTok{ pos, }\AttributeTok{cex =} \FloatTok{0.8}\NormalTok{, }\AttributeTok{col =} \StringTok{"blue"}\NormalTok{, }\AttributeTok{offset =} \FloatTok{0.4}\NormalTok{)}
\end{Highlighting}
\end{Shaded}

\pandocbounded{\includegraphics[keepaspectratio]{WilloughbyHW4_files/figure-latex/unnamed-chunk-8-1.pdf}}

Next, I assess whether any of the (above) candidate observations are
formally outliers. For each observation \(i\), I compare \(|t_i|\) with
\(t^{\alpha/2}_{n - (p + 1) - 1}\). Following this test, observation 23
is the only point deemed an outlier at the \(\alpha = 0.05\)
significance level. This is shown in the below plot.

\begin{Shaded}
\begin{Highlighting}[]
\NormalTok{ti }\OtherTok{\textless{}{-}} \FunctionTok{rstudent}\NormalTok{(fit2)}
\NormalTok{n }\OtherTok{\textless{}{-}} \FunctionTok{length}\NormalTok{(ti)}
\NormalTok{df\_t }\OtherTok{\textless{}{-}} \FunctionTok{df.residual}\NormalTok{(fit2) }\SpecialCharTok{{-}} \DecValTok{1} \CommentTok{\# df for externally studentized residuals}
\NormalTok{alpha }\OtherTok{\textless{}{-}} \FloatTok{0.05} \CommentTok{\# default significance level }

\CommentTok{\# two{-}sided p{-}values and significant indices}
\NormalTok{pval }\OtherTok{\textless{}{-}} \DecValTok{2} \SpecialCharTok{*} \FunctionTok{pt}\NormalTok{(}\SpecialCharTok{{-}}\FunctionTok{abs}\NormalTok{(ti), }\AttributeTok{df =}\NormalTok{ df\_t)}
\NormalTok{sig\_idx }\OtherTok{\textless{}{-}} \FunctionTok{which}\NormalTok{(pval }\SpecialCharTok{\textless{}}\NormalTok{ alpha)}

\CommentTok{\# critical value for plotting reference lines}
\NormalTok{crit }\OtherTok{\textless{}{-}} \FunctionTok{qt}\NormalTok{(}\DecValTok{1} \SpecialCharTok{{-}}\NormalTok{ alpha}\SpecialCharTok{/}\DecValTok{2}\NormalTok{, }\AttributeTok{df =}\NormalTok{ df\_t)}

\FunctionTok{plot}\NormalTok{(}
\NormalTok{  ti,}
  \AttributeTok{pch =} \DecValTok{19}\NormalTok{,}
  \AttributeTok{col =} \FunctionTok{ifelse}\NormalTok{(}\FunctionTok{seq\_along}\NormalTok{(ti) }\SpecialCharTok{\%in\%}\NormalTok{ sig\_idx, }\StringTok{"blue"}\NormalTok{, }\StringTok{"gray30"}\NormalTok{),}
  \AttributeTok{main =} \StringTok{"Externally Studentized Residual"}\NormalTok{, }
  \AttributeTok{xlab =} \StringTok{"Observation index"}\NormalTok{,}
  \AttributeTok{ylab =} \StringTok{"Externally studentized residual"}\NormalTok{,}
  \AttributeTok{ylim =} \FunctionTok{range}\NormalTok{(}\FunctionTok{c}\NormalTok{(ti, }\SpecialCharTok{{-}}\NormalTok{crit }\SpecialCharTok{{-}} \DecValTok{1}\NormalTok{, crit }\SpecialCharTok{+} \DecValTok{1}\NormalTok{))}
\NormalTok{)}
\FunctionTok{abline}\NormalTok{(}\AttributeTok{h =} \FunctionTok{c}\NormalTok{(}\SpecialCharTok{{-}}\NormalTok{crit, crit), }\AttributeTok{col =} \StringTok{"red"}\NormalTok{, }\AttributeTok{lty =} \DecValTok{2}\NormalTok{, }\AttributeTok{lwd =} \DecValTok{1}\NormalTok{)}

\CommentTok{\# label only significant points}
\ControlFlowTok{if}\NormalTok{ (}\FunctionTok{length}\NormalTok{(sig\_idx)) \{}
\NormalTok{  pos }\OtherTok{\textless{}{-}} \FunctionTok{ifelse}\NormalTok{(ti[sig\_idx] }\SpecialCharTok{\textgreater{}=} \DecValTok{0}\NormalTok{, }\DecValTok{3}\NormalTok{, }\DecValTok{1}\NormalTok{)}
  \FunctionTok{text}\NormalTok{(sig\_idx, ti[sig\_idx], }\AttributeTok{labels =}\NormalTok{ sig\_idx, }\AttributeTok{pos =}\NormalTok{ pos, }\AttributeTok{cex =} \FloatTok{0.8}\NormalTok{, }\AttributeTok{col =} \StringTok{"blue"}\NormalTok{, }\AttributeTok{offset =} \FloatTok{0.4}\NormalTok{)}
\NormalTok{\}}
\end{Highlighting}
\end{Shaded}

\pandocbounded{\includegraphics[keepaspectratio]{WilloughbyHW4_files/figure-latex/unnamed-chunk-9-1.pdf}}

While conducting the outlier detection test for all observations
simultaneously, in practice, we will reject too many points.
Furthermore:

\[\text{ Type I Error } = P_{H_o}(\text{reject at least one test)} \\ \le \sum_iP_{H_o}(\text{reject test i}) \\ = n\alpha\]
Thus, the Bonferroni correction appropriately controls for the Type I
Error by testing each hypothesis at the \(\alpha/n\) significance level.
In doing so, observation 23 is no longer deemed an outlier. In fact, the
critical value threshold is too large for any point to be deemed an
outlier under the fitted model. While this provides contradictory
results, I have shown the implication of not properly controlling for
the Type I Error Rate.

\begin{Shaded}
\begin{Highlighting}[]
\NormalTok{ti }\OtherTok{\textless{}{-}} \FunctionTok{rstudent}\NormalTok{(fit2)}
\NormalTok{n }\OtherTok{\textless{}{-}} \FunctionTok{length}\NormalTok{(ti)}
\NormalTok{df\_t }\OtherTok{\textless{}{-}} \FunctionTok{df.residual}\NormalTok{(fit2) }\SpecialCharTok{{-}} \DecValTok{1} \CommentTok{\# df for externally studentized residuals}
\NormalTok{alpha }\OtherTok{\textless{}{-}} \FloatTok{0.05}
\NormalTok{alpha\_bonf }\OtherTok{\textless{}{-}}\NormalTok{ alpha }\SpecialCharTok{/}\NormalTok{ n}

\CommentTok{\# two{-}sided p{-}values and significant indices}
\NormalTok{pval }\OtherTok{\textless{}{-}} \DecValTok{2} \SpecialCharTok{*} \FunctionTok{pt}\NormalTok{(}\SpecialCharTok{{-}}\FunctionTok{abs}\NormalTok{(ti), }\AttributeTok{df =}\NormalTok{ df\_t)}
\NormalTok{sig\_idx }\OtherTok{\textless{}{-}} \FunctionTok{which}\NormalTok{(pval }\SpecialCharTok{\textless{}}\NormalTok{ alpha\_bonf)}

\CommentTok{\# critical value for plotting reference lines}
\NormalTok{crit }\OtherTok{\textless{}{-}} \FunctionTok{qt}\NormalTok{(}\DecValTok{1} \SpecialCharTok{{-}}\NormalTok{ alpha\_bonf}\SpecialCharTok{/}\DecValTok{2}\NormalTok{, }\AttributeTok{df =}\NormalTok{ df\_t)}

\FunctionTok{plot}\NormalTok{(}
\NormalTok{  ti,}
  \AttributeTok{pch =} \DecValTok{19}\NormalTok{,}
  \AttributeTok{col =} \FunctionTok{ifelse}\NormalTok{(}\FunctionTok{seq\_along}\NormalTok{(ti) }\SpecialCharTok{\%in\%}\NormalTok{ sig\_idx, }\StringTok{"blue"}\NormalTok{, }\StringTok{"gray30"}\NormalTok{),}
  \AttributeTok{main =} \FunctionTok{sprintf}\NormalTok{(}\StringTok{"Externally Studentized Residuals (Bonferroni α/n = \%.4f)"}\NormalTok{, alpha\_bonf),}
  \AttributeTok{xlab =} \StringTok{"Observation index"}\NormalTok{,}
  \AttributeTok{ylab =} \StringTok{"Externally studentized residual"}\NormalTok{,}
  \AttributeTok{ylim =} \FunctionTok{range}\NormalTok{(}\FunctionTok{c}\NormalTok{(ti, }\SpecialCharTok{{-}}\NormalTok{crit, crit))}
\NormalTok{)}
\FunctionTok{abline}\NormalTok{(}\AttributeTok{h =} \FunctionTok{c}\NormalTok{(}\SpecialCharTok{{-}}\NormalTok{crit, crit), }\AttributeTok{col =} \StringTok{"red"}\NormalTok{, }\AttributeTok{lty =} \DecValTok{2}\NormalTok{, }\AttributeTok{lwd =} \DecValTok{1}\NormalTok{)}
\end{Highlighting}
\end{Shaded}

\pandocbounded{\includegraphics[keepaspectratio]{WilloughbyHW4_files/figure-latex/unnamed-chunk-10-1.pdf}}

\begin{Shaded}
\begin{Highlighting}[]
\CommentTok{\# label only significant points}
\ControlFlowTok{if}\NormalTok{ (}\FunctionTok{length}\NormalTok{(sig\_idx)) \{}
\NormalTok{  pos }\OtherTok{\textless{}{-}} \FunctionTok{ifelse}\NormalTok{(ti[sig\_idx] }\SpecialCharTok{\textgreater{}=} \DecValTok{0}\NormalTok{, }\DecValTok{3}\NormalTok{, }\DecValTok{1}\NormalTok{)}
  \FunctionTok{text}\NormalTok{(sig\_idx, ti[sig\_idx], }\AttributeTok{labels =}\NormalTok{ sig\_idx, }\AttributeTok{pos =}\NormalTok{ pos, }\AttributeTok{cex =} \FloatTok{0.8}\NormalTok{, }\AttributeTok{col =} \StringTok{"blue"}\NormalTok{, }\AttributeTok{offset =} \FloatTok{0.4}\NormalTok{)}
\NormalTok{\}}
\end{Highlighting}
\end{Shaded}


\end{document}
