% Options for packages loaded elsewhere
\PassOptionsToPackage{unicode}{hyperref}
\PassOptionsToPackage{hyphens}{url}
\documentclass[
]{article}
\usepackage{xcolor}
\usepackage[margin=1in]{geometry}
\usepackage{amsmath,amssymb}
\setcounter{secnumdepth}{-\maxdimen} % remove section numbering
\usepackage{iftex}
\ifPDFTeX
  \usepackage[T1]{fontenc}
  \usepackage[utf8]{inputenc}
  \usepackage{textcomp} % provide euro and other symbols
\else % if luatex or xetex
  \usepackage{unicode-math} % this also loads fontspec
  \defaultfontfeatures{Scale=MatchLowercase}
  \defaultfontfeatures[\rmfamily]{Ligatures=TeX,Scale=1}
\fi
\usepackage{lmodern}
\ifPDFTeX\else
  % xetex/luatex font selection
\fi
% Use upquote if available, for straight quotes in verbatim environments
\IfFileExists{upquote.sty}{\usepackage{upquote}}{}
\IfFileExists{microtype.sty}{% use microtype if available
  \usepackage[]{microtype}
  \UseMicrotypeSet[protrusion]{basicmath} % disable protrusion for tt fonts
}{}
\makeatletter
\@ifundefined{KOMAClassName}{% if non-KOMA class
  \IfFileExists{parskip.sty}{%
    \usepackage{parskip}
  }{% else
    \setlength{\parindent}{0pt}
    \setlength{\parskip}{6pt plus 2pt minus 1pt}}
}{% if KOMA class
  \KOMAoptions{parskip=half}}
\makeatother
\usepackage{color}
\usepackage{fancyvrb}
\newcommand{\VerbBar}{|}
\newcommand{\VERB}{\Verb[commandchars=\\\{\}]}
\DefineVerbatimEnvironment{Highlighting}{Verbatim}{commandchars=\\\{\}}
% Add ',fontsize=\small' for more characters per line
\usepackage{framed}
\definecolor{shadecolor}{RGB}{248,248,248}
\newenvironment{Shaded}{\begin{snugshade}}{\end{snugshade}}
\newcommand{\AlertTok}[1]{\textcolor[rgb]{0.94,0.16,0.16}{#1}}
\newcommand{\AnnotationTok}[1]{\textcolor[rgb]{0.56,0.35,0.01}{\textbf{\textit{#1}}}}
\newcommand{\AttributeTok}[1]{\textcolor[rgb]{0.13,0.29,0.53}{#1}}
\newcommand{\BaseNTok}[1]{\textcolor[rgb]{0.00,0.00,0.81}{#1}}
\newcommand{\BuiltInTok}[1]{#1}
\newcommand{\CharTok}[1]{\textcolor[rgb]{0.31,0.60,0.02}{#1}}
\newcommand{\CommentTok}[1]{\textcolor[rgb]{0.56,0.35,0.01}{\textit{#1}}}
\newcommand{\CommentVarTok}[1]{\textcolor[rgb]{0.56,0.35,0.01}{\textbf{\textit{#1}}}}
\newcommand{\ConstantTok}[1]{\textcolor[rgb]{0.56,0.35,0.01}{#1}}
\newcommand{\ControlFlowTok}[1]{\textcolor[rgb]{0.13,0.29,0.53}{\textbf{#1}}}
\newcommand{\DataTypeTok}[1]{\textcolor[rgb]{0.13,0.29,0.53}{#1}}
\newcommand{\DecValTok}[1]{\textcolor[rgb]{0.00,0.00,0.81}{#1}}
\newcommand{\DocumentationTok}[1]{\textcolor[rgb]{0.56,0.35,0.01}{\textbf{\textit{#1}}}}
\newcommand{\ErrorTok}[1]{\textcolor[rgb]{0.64,0.00,0.00}{\textbf{#1}}}
\newcommand{\ExtensionTok}[1]{#1}
\newcommand{\FloatTok}[1]{\textcolor[rgb]{0.00,0.00,0.81}{#1}}
\newcommand{\FunctionTok}[1]{\textcolor[rgb]{0.13,0.29,0.53}{\textbf{#1}}}
\newcommand{\ImportTok}[1]{#1}
\newcommand{\InformationTok}[1]{\textcolor[rgb]{0.56,0.35,0.01}{\textbf{\textit{#1}}}}
\newcommand{\KeywordTok}[1]{\textcolor[rgb]{0.13,0.29,0.53}{\textbf{#1}}}
\newcommand{\NormalTok}[1]{#1}
\newcommand{\OperatorTok}[1]{\textcolor[rgb]{0.81,0.36,0.00}{\textbf{#1}}}
\newcommand{\OtherTok}[1]{\textcolor[rgb]{0.56,0.35,0.01}{#1}}
\newcommand{\PreprocessorTok}[1]{\textcolor[rgb]{0.56,0.35,0.01}{\textit{#1}}}
\newcommand{\RegionMarkerTok}[1]{#1}
\newcommand{\SpecialCharTok}[1]{\textcolor[rgb]{0.81,0.36,0.00}{\textbf{#1}}}
\newcommand{\SpecialStringTok}[1]{\textcolor[rgb]{0.31,0.60,0.02}{#1}}
\newcommand{\StringTok}[1]{\textcolor[rgb]{0.31,0.60,0.02}{#1}}
\newcommand{\VariableTok}[1]{\textcolor[rgb]{0.00,0.00,0.00}{#1}}
\newcommand{\VerbatimStringTok}[1]{\textcolor[rgb]{0.31,0.60,0.02}{#1}}
\newcommand{\WarningTok}[1]{\textcolor[rgb]{0.56,0.35,0.01}{\textbf{\textit{#1}}}}
\usepackage{graphicx}
\makeatletter
\newsavebox\pandoc@box
\newcommand*\pandocbounded[1]{% scales image to fit in text height/width
  \sbox\pandoc@box{#1}%
  \Gscale@div\@tempa{\textheight}{\dimexpr\ht\pandoc@box+\dp\pandoc@box\relax}%
  \Gscale@div\@tempb{\linewidth}{\wd\pandoc@box}%
  \ifdim\@tempb\p@<\@tempa\p@\let\@tempa\@tempb\fi% select the smaller of both
  \ifdim\@tempa\p@<\p@\scalebox{\@tempa}{\usebox\pandoc@box}%
  \else\usebox{\pandoc@box}%
  \fi%
}
% Set default figure placement to htbp
\def\fps@figure{htbp}
\makeatother
\setlength{\emergencystretch}{3em} % prevent overfull lines
\providecommand{\tightlist}{%
  \setlength{\itemsep}{0pt}\setlength{\parskip}{0pt}}
\usepackage{bookmark}
\IfFileExists{xurl.sty}{\usepackage{xurl}}{} % add URL line breaks if available
\urlstyle{same}
\hypersetup{
  pdftitle={WilloughbyHW6},
  pdfauthor={Bryant Willoughby},
  hidelinks,
  pdfcreator={LaTeX via pandoc}}

\title{WilloughbyHW6}
\author{Bryant Willoughby}
\date{2025-10-24}

\begin{document}
\maketitle

\section{Problem 1}\label{problem-1}

Description: In a study of cheddar cheese, samples of cheese were
analyzed for their chemical composition and were subjected to taste
tests. Overall taste scores were obtained by combining the scores from
several tasters. This is a data frame with 30 observations on four
variables.

\begin{Shaded}
\begin{Highlighting}[]
\CommentTok{\# help(cheddar)}
\FunctionTok{data}\NormalTok{(cheddar)}
\FunctionTok{head}\NormalTok{(cheddar)}
\end{Highlighting}
\end{Shaded}

\begin{verbatim}
##   taste Acetic   H2S Lactic
## 1  12.3  4.543 3.135   0.86
## 2  20.9  5.159 5.043   1.53
## 3  39.0  5.366 5.438   1.57
## 4  47.9  5.759 7.496   1.81
## 5   5.6  4.663 3.807   0.99
## 6  25.9  5.697 7.601   1.09
\end{verbatim}

Using the \texttt{cheddar} data, fit a linear model with \texttt{taste}
as the response and the other three variables as predictors.

\begin{Shaded}
\begin{Highlighting}[]
\NormalTok{model.OLS }\OtherTok{\textless{}{-}} \FunctionTok{lm}\NormalTok{(taste }\SpecialCharTok{\textasciitilde{}}\NormalTok{ ., }\AttributeTok{data =}\NormalTok{ cheddar)}
\FunctionTok{summary}\NormalTok{(model.OLS)}
\end{Highlighting}
\end{Shaded}

\begin{verbatim}
## 
## Call:
## lm(formula = taste ~ ., data = cheddar)
## 
## Residuals:
##     Min      1Q  Median      3Q     Max 
## -17.390  -6.612  -1.009   4.908  25.449 
## 
## Coefficients:
##             Estimate Std. Error t value Pr(>|t|)   
## (Intercept) -28.8768    19.7354  -1.463  0.15540   
## Acetic        0.3277     4.4598   0.073  0.94198   
## H2S           3.9118     1.2484   3.133  0.00425 **
## Lactic       19.6705     8.6291   2.280  0.03108 * 
## ---
## Signif. codes:  0 '***' 0.001 '**' 0.01 '*' 0.05 '.' 0.1 ' ' 1
## 
## Residual standard error: 10.13 on 26 degrees of freedom
## Multiple R-squared:  0.6518, Adjusted R-squared:  0.6116 
## F-statistic: 16.22 on 3 and 26 DF,  p-value: 3.81e-06
\end{verbatim}

\begin{itemize}
\tightlist
\item
  Suppose that the observations were taken in time order. Create a time
  variable. Plot the residuals of the model against time and comment on
  what can be seen.
\end{itemize}

\begin{Shaded}
\begin{Highlighting}[]
\NormalTok{time.var }\OtherTok{\textless{}{-}} \DecValTok{1}\SpecialCharTok{:}\FunctionTok{nrow}\NormalTok{(cheddar)}
\NormalTok{cheddar }\OtherTok{\textless{}{-}} \FunctionTok{cbind}\NormalTok{(cheddar, time.var)}
\FunctionTok{head}\NormalTok{(cheddar)}
\end{Highlighting}
\end{Shaded}

\begin{verbatim}
##   taste Acetic   H2S Lactic time.var
## 1  12.3  4.543 3.135   0.86        1
## 2  20.9  5.159 5.043   1.53        2
## 3  39.0  5.366 5.438   1.57        3
## 4  47.9  5.759 7.496   1.81        4
## 5   5.6  4.663 3.807   0.99        5
## 6  25.9  5.697 7.601   1.09        6
\end{verbatim}

Notice how the residuals are decreasing linearly over time. This gives
visual (informal) evidence to suggest that the errors (residuals) are
correlated over time, violating one of the OLS assumptions.

\begin{Shaded}
\begin{Highlighting}[]
\FunctionTok{plot}\NormalTok{(time.var, model.OLS}\SpecialCharTok{$}\NormalTok{residuals,}
     \AttributeTok{main =} \StringTok{"Cheddar: OLS Residuals vs Time"}\NormalTok{,}
     \AttributeTok{xlab =} \StringTok{"Time (index)"}\NormalTok{,}
     \AttributeTok{ylab =} \StringTok{"Residuals"}\NormalTok{)}
\FunctionTok{abline}\NormalTok{(}\FunctionTok{lm}\NormalTok{(}\FunctionTok{resid}\NormalTok{(model.OLS) }\SpecialCharTok{\textasciitilde{}}\NormalTok{ time.var, }\AttributeTok{data =}\NormalTok{ cheddar), }\AttributeTok{col =} \StringTok{"red"}\NormalTok{, }\AttributeTok{lwd =} \DecValTok{2}\NormalTok{)}
\end{Highlighting}
\end{Shaded}

\pandocbounded{\includegraphics[keepaspectratio]{WilloughbyHW6_files/figure-latex/unnamed-chunk-4-1.pdf}}

\begin{itemize}
\tightlist
\item
  Fit a GLS model with same form as above but now allow for an AR(1)
  correlation among the errors. Is there evidence of such a correlation?
\end{itemize}

\begin{Shaded}
\begin{Highlighting}[]
\NormalTok{model.GLS }\OtherTok{\textless{}{-}} \FunctionTok{gls}\NormalTok{(taste }\SpecialCharTok{\textasciitilde{}}\NormalTok{ Acetic }\SpecialCharTok{+}\NormalTok{ H2S }\SpecialCharTok{+}\NormalTok{ Lactic, }
                 \AttributeTok{correlation =} \FunctionTok{corAR1}\NormalTok{(}\AttributeTok{form =} \SpecialCharTok{\textasciitilde{}}\NormalTok{ time.var),}
                 \AttributeTok{data =}\NormalTok{ cheddar)}
\FunctionTok{summary}\NormalTok{(model.GLS)}
\end{Highlighting}
\end{Shaded}

\begin{verbatim}
## Generalized least squares fit by REML
##   Model: taste ~ Acetic + H2S + Lactic 
##   Data: cheddar 
##      AIC      BIC  logLik
##   214.94 222.4886 -101.47
## 
## Correlation Structure: AR(1)
##  Formula: ~time.var 
##  Parameter estimate(s):
##       Phi 
## 0.2641944 
## 
## Coefficients:
##                  Value Std.Error   t-value p-value
## (Intercept) -30.332472 20.273077 -1.496195  0.1466
## Acetic        1.436411  4.876581  0.294553  0.7707
## H2S           4.058880  1.314283  3.088284  0.0047
## Lactic       15.826468  9.235404  1.713674  0.0985
## 
##  Correlation: 
##        (Intr) Acetic H2S   
## Acetic -0.899              
## H2S     0.424 -0.395       
## Lactic  0.063 -0.416 -0.435
## 
## Standardized residuals:
##         Min          Q1         Med          Q3         Max 
## -1.64546468 -0.63861716 -0.06641714  0.52255676  2.41323020 
## 
## Residual standard error: 10.33276 
## Degrees of freedom: 30 total; 26 residual
\end{verbatim}

Under the autoregressive model as seen above, \(\rho\) is used to test
correlated errors. Formally, we are testing
\(H_o: \rho = 0 \text{ vs } H_a: \rho \ne 0\). Assuming this test uses
the \(\alpha = 0.05\) significance level, we can use a
\(100(1 - (\alpha = 0.05)) = 95\%\) confidence level to address the same
statistical question. That is, the below \(95\%\) CI for \(\rho\)
contains zero. Thus, we fail to reject \(H_o\) and there is not
significant evidence to suggest the errors are correlated.

\begin{Shaded}
\begin{Highlighting}[]
\FunctionTok{intervals}\NormalTok{(model.GLS)}
\end{Highlighting}
\end{Shaded}

\begin{verbatim}
## Approximate 95% confidence intervals
## 
##  Coefficients:
##                  lower       est.     upper
## (Intercept) -72.004379 -30.332472 11.339436
## Acetic       -8.587544   1.436411 11.460367
## H2S           1.357332   4.058880  6.760427
## Lactic       -3.157177  15.826468 34.810113
## 
##  Correlation structure:
##          lower      est.     upper
## Phi -0.1691951 0.2641944 0.6119685
## 
##  Residual standard error:
##     lower      est.     upper 
##  7.626295 10.332756 13.999701
\end{verbatim}

\begin{itemize}
\tightlist
\item
  Fit a OLS model but now with time as an additional predictor.
  Investigate the significance of time in the model.
\end{itemize}

In the following OLS model, \texttt{H2S} and \texttt{Lactic} are
significant predictors, as was seen in the original OLS model above.
Now, the manually constructed \texttt{time.var} is also significant in
this model. That suggests there is an additional time component that is
missing in the original OLS model.

\begin{Shaded}
\begin{Highlighting}[]
\NormalTok{model.OLS.time }\OtherTok{\textless{}{-}} \FunctionTok{lm}\NormalTok{(taste }\SpecialCharTok{\textasciitilde{}}\NormalTok{ ., }\AttributeTok{data =}\NormalTok{ cheddar)}
\FunctionTok{summary}\NormalTok{(model.OLS.time)}
\end{Highlighting}
\end{Shaded}

\begin{verbatim}
## 
## Call:
## lm(formula = taste ~ ., data = cheddar)
## 
## Residuals:
##      Min       1Q   Median       3Q      Max 
## -22.3523  -4.9735  -0.5089   4.8531  23.1311 
## 
## Coefficients:
##             Estimate Std. Error t value Pr(>|t|)   
## (Intercept) -36.6127    17.9845  -2.036  0.05250 . 
## Acetic        4.1275     4.2556   0.970  0.34139   
## H2S           3.5387     1.1315   3.127  0.00444 **
## Lactic       17.9527     7.7875   2.305  0.02973 * 
## time.var     -0.5459     0.2043  -2.672  0.01306 * 
## ---
## Signif. codes:  0 '***' 0.001 '**' 0.01 '*' 0.05 '.' 0.1 ' ' 1
## 
## Residual standard error: 9.112 on 25 degrees of freedom
## Multiple R-squared:  0.7291, Adjusted R-squared:  0.6858 
## F-statistic: 16.83 on 4 and 25 DF,  p-value: 8.205e-07
\end{verbatim}

\begin{itemize}
\tightlist
\item
  The last two models have both allowed for an effect of time. Explain
  how they do this differently.
\end{itemize}

The GLS model treats time as a source of autocorrelation in the
residuals, whereas the OLS model treats time as an explicit predictor
influencing the mean response. Since these reflect different mechanisms,
it is possible to find a significant time trend in the mean (OLS + time)
even when the estimated residual correlation parameter \(\rho\) from the
GLS model is not significant. This suggests that the apparent time
pattern in the residual plot likely reflects a systematic trend in taste
over time rather than serially correlated random fluctuations (errors
depend on previous errors even after controlling for predictors).

\section{Problem 2}\label{problem-2}

Using the \texttt{sat} data, fit a model with \texttt{total} as the
response and \texttt{takers}, \texttt{ratio}, \texttt{salary} and
\texttt{expend} as predictors using the following methods:

Description: The \texttt{sat} data frame has 50 rows and 7 columns. Data
were collected to study the relationship between expenditures on public
education and test results.

\begin{Shaded}
\begin{Highlighting}[]
\CommentTok{\# help(sat)}
\FunctionTok{data}\NormalTok{(sat)}
\FunctionTok{head}\NormalTok{(sat)}
\end{Highlighting}
\end{Shaded}

\begin{verbatim}
##            expend ratio salary takers verbal math total
## Alabama     4.405  17.2 31.144      8    491  538  1029
## Alaska      8.963  17.6 47.951     47    445  489   934
## Arizona     4.778  19.3 32.175     27    448  496   944
## Arkansas    4.459  17.1 28.934      6    482  523  1005
## California  4.992  24.0 41.078     45    417  485   902
## Colorado    5.443  18.4 34.571     29    462  518   980
\end{verbatim}

\begin{itemize}
\tightlist
\item
  Ordinary least squares
\end{itemize}

I consult the p-values here to assess significance of predictors in the
model.

\begin{Shaded}
\begin{Highlighting}[]
\NormalTok{model.OLS }\OtherTok{\textless{}{-}} \FunctionTok{lm}\NormalTok{(total }\SpecialCharTok{\textasciitilde{}}\NormalTok{ takers }\SpecialCharTok{+}\NormalTok{ ratio }\SpecialCharTok{+}\NormalTok{ salary }\SpecialCharTok{+}\NormalTok{ expend, }\AttributeTok{data =}\NormalTok{ sat)}
\FunctionTok{summary}\NormalTok{(model.OLS)}
\end{Highlighting}
\end{Shaded}

\begin{verbatim}
## 
## Call:
## lm(formula = total ~ takers + ratio + salary + expend, data = sat)
## 
## Residuals:
##     Min      1Q  Median      3Q     Max 
## -90.531 -20.855  -1.746  15.979  66.571 
## 
## Coefficients:
##              Estimate Std. Error t value Pr(>|t|)    
## (Intercept) 1045.9715    52.8698  19.784  < 2e-16 ***
## takers        -2.9045     0.2313 -12.559 2.61e-16 ***
## ratio         -3.6242     3.2154  -1.127    0.266    
## salary         1.6379     2.3872   0.686    0.496    
## expend         4.4626    10.5465   0.423    0.674    
## ---
## Signif. codes:  0 '***' 0.001 '**' 0.01 '*' 0.05 '.' 0.1 ' ' 1
## 
## Residual standard error: 32.7 on 45 degrees of freedom
## Multiple R-squared:  0.8246, Adjusted R-squared:  0.809 
## F-statistic: 52.88 on 4 and 45 DF,  p-value: < 2.2e-16
\end{verbatim}

\begin{itemize}
\tightlist
\item
  Least absolute deviations
\end{itemize}

I consult the confidence intervals to assess significance in the
predictors. Since the same significance level (\(\alpha = 0.05\)) is
being used, a CI excluding zero provides evidence for a signifcant
predictor.

\begin{Shaded}
\begin{Highlighting}[]
\NormalTok{model.LAD }\OtherTok{\textless{}{-}} \FunctionTok{rq}\NormalTok{(total }\SpecialCharTok{\textasciitilde{}}\NormalTok{ takers }\SpecialCharTok{+}\NormalTok{ ratio }\SpecialCharTok{+}\NormalTok{ salary }\SpecialCharTok{+}\NormalTok{ expend, }\AttributeTok{data =}\NormalTok{ sat)}
\FunctionTok{summary}\NormalTok{(model.LAD)}
\end{Highlighting}
\end{Shaded}

\begin{verbatim}
## 
## Call: rq(formula = total ~ takers + ratio + salary + expend, data = sat)
## 
## tau: [1] 0.5
## 
## Coefficients:
##             coefficients lower bd   upper bd  
## (Intercept) 1090.89886    920.17149 1151.85075
## takers        -3.13961     -3.38485   -2.66479
## ratio         -7.26632    -10.73796    1.62341
## salary         3.18313     -0.15788    5.41909
## expend        -0.79753     -8.88001   20.92522
\end{verbatim}

\begin{itemize}
\tightlist
\item
  Huber's robust regression
\end{itemize}

\begin{Shaded}
\begin{Highlighting}[]
\FunctionTok{round}\NormalTok{(}\FunctionTok{qt}\NormalTok{(}\FloatTok{0.025}\NormalTok{, }\AttributeTok{df =} \DecValTok{45}\NormalTok{, }\AttributeTok{lower.tail =}\NormalTok{ F),}\DecValTok{3}\NormalTok{)}
\end{Highlighting}
\end{Shaded}

\begin{verbatim}
## [1] 2.014
\end{verbatim}

A predictor is significant if its respective t-test statistics has the
following property: \(|t| \lt 2.014\)

\begin{Shaded}
\begin{Highlighting}[]
\NormalTok{model.HM }\OtherTok{\textless{}{-}} \FunctionTok{rlm}\NormalTok{(total }\SpecialCharTok{\textasciitilde{}}\NormalTok{ takers }\SpecialCharTok{+}\NormalTok{ ratio }\SpecialCharTok{+}\NormalTok{ salary }\SpecialCharTok{+}\NormalTok{ expend, }\AttributeTok{data =}\NormalTok{ sat)}
\FunctionTok{summary}\NormalTok{(model.HM)}
\end{Highlighting}
\end{Shaded}

\begin{verbatim}
## 
## Call: rlm(formula = total ~ takers + ratio + salary + expend, data = sat)
## Residuals:
##     Min      1Q  Median      3Q     Max 
## -92.510 -17.701  -1.002  15.015  77.058 
## 
## Coefficients:
##             Value     Std. Error t value  
## (Intercept) 1060.2074   49.8845    21.2533
## takers        -2.9778    0.2182   -13.6470
## ratio         -5.1254    3.0339    -1.6894
## salary         2.0933    2.2525     0.9293
## expend         3.9158    9.9510     0.3935
## 
## Residual standard error: 25.58 on 45 degrees of freedom
\end{verbatim}

Compare the results. In each case, comment on the significance of
predictors.

Across all three regression methods: ordinary least squares (OLS), least
absolute deviations (LAD), and Huber's robust regression,
\texttt{takers} emerges as a consistently significant predictor of total
SAT score at the \(\alpha=0.05\). None of the other predictors reach
statistical significance across all models, suggesting that the
proportion of test takers is the most stable and influential variable in
explaining variation in SAT performance.

The sign reversal of the \texttt{expend} coefficient in the LAD model
may suggest that a few high-expenditure states exert substantial
influence in the OLS and Huber fits, pulling their coefficients
positive. Because LAD regression minimizes absolute rather than squared
residuals, it is less sensitive to such extreme values, so its negative
coefficient could reflect the trend among the majority of observations
rather than the influence of outliers. However, this interpretation
should be made cautiously; additional diagnostic plots (e.g., residual
vs.~leverage, influence measures) would be needed to confirm whether
specific high-expenditure states are truly driving the difference.

\begin{Shaded}
\begin{Highlighting}[]
\FunctionTok{coef}\NormalTok{(model.OLS)}
\end{Highlighting}
\end{Shaded}

\begin{verbatim}
## (Intercept)      takers       ratio      salary      expend 
## 1045.971536   -2.904481   -3.624232    1.637917    4.462594
\end{verbatim}

\begin{Shaded}
\begin{Highlighting}[]
\FunctionTok{coef}\NormalTok{(model.LAD)}
\end{Highlighting}
\end{Shaded}

\begin{verbatim}
##  (Intercept)       takers        ratio       salary       expend 
## 1090.8988638   -3.1396146   -7.2663187    3.1831325   -0.7975319
\end{verbatim}

\begin{Shaded}
\begin{Highlighting}[]
\FunctionTok{coef}\NormalTok{(model.HM)}
\end{Highlighting}
\end{Shaded}

\begin{verbatim}
## (Intercept)      takers       ratio      salary      expend 
## 1060.207357   -2.977805   -5.125365    2.093258    3.915810
\end{verbatim}

\end{document}
